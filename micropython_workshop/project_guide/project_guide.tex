% Some help here: https://en.wikibooks.org/wiki/LaTeX/Document_Structure

% Preamble
% ---
\documentclass{report}

% Packages
% ---
\usepackage{graphicx} % Add pictures to your document
\graphicspath{ {images/} }  % we will keep our images ogranized under the images directory
\usepackage{listings} % Source code formatting and highlighting
\usepackage{xcolor}   % needed to color hyperlinks and source code
\usepackage{hyperref} % create hyperlinks
\usepackage[T1]{fontenc} % make sure quotes in Python listings are straight quotes
\usepackage{tcolorbox}
\usepackage{siunitx} % Allows us to use math symbols (like Omega) in normal paragraphs

\hypersetup{
    colorlinks=true, %set true if you want colored links
    linktoc=all,     %set to all if you want both sections and subsections linked
    linkcolor=blue,  %choose some color if you want links to stand out
}

\lstloadlanguages{Python}
\lstset{
  language=Python,
  numberstyle=\color{gray},
  stringstyle=\color[HTML]{933797},
  commentstyle=\color[HTML]{228B22},
  emph={[2]from,import,pass,return}, emphstyle={[2]\color[HTML]{DD52F0}},
  emph={[3]range}, emphstyle={[3]\color[HTML]{D17032}},
  emph={[4]for,in,def}, emphstyle={[4]\color{blue}},
  showstringspaces=false,
  breaklines=true,
  prebreak=\mbox{{\color{gray}\tiny$\searrow$}},
  numbers=left,
  xleftmargin=15pt,
}

\begin{document}

\title{MicroPython and Microcontrollers}
\author{NetApp YWIT}
\date{May 1, 2020}
\maketitle

\tableofcontents

\chapter{Introduction}
This workshop will introduce the student to Python coding, electronics, and project
design. We will building several projects ranging from simple to more complex. These
projects are based on the ESP32-C3 microcontroller which is running MicroPython and
they depend on some other electronics components such as LEDs, buttons, and more.

Students are encouraged to make use of the written instructions, diagrams, and to
explore and play with the components for themselves to see how they work and what
they do. There are many online resources as well for learning.

Connecting the microcontroller to your computer may require the installation of a
USB driver. If so, download the driver for your OS from this page and install as instructed:
https://ftdichip.com/drivers/vcp-drivers/. Once installed successfully, unplug and replug
the USB cable into the microcontroller and then press the small reset button on the
microcontroller board labeled "R" (for reset).


\chapter{Project 1: Blink}
% chapter contents
\input{chapter_2/chapter_2.tex}
\chapter{Project 3: LED Party}
% chapter contents
\chapter{Project 4: Sensor}

\section{Overview}
Now that you have a handle on how to work with LEDs and buttons, let's look at something a little more complex.
This project will introduce a humidity and temperature sensor. Over the course of this project, you will:
\begin{itemize}
    \item Connect a sensor board to your microcontroller
    \item Import a library into your code that knows how to communicate with that sensor
    \item Use MicroPython to poll the sensor over and over to write out the current result
    \item Connect a screen to your microcontroller and output the sensor data to it
\end{itemize}
At the end of this project, your microcontroller should run a MicroPython program that prints out the current
temperature and humidity of the room you're sitting in. Let's get started!
\begin{figure}[H]
\centering
    \includegraphics[width=.6\linewidth]{chapter_4/screen_connected.jpg}
    \caption{The end result should look something like this}
\end{figure}

\pagebreak

\section{Directions}

\subsection{Creating the circuit}
Using jumper cables, you will be assembling a circuit between your microcontroller, your breadboard, and the small
temperature/humidity board included in your kit.

\subsubsection{Remove previous components}
Before beginning, remove any components from prior chapters including LEDs, buttons, and wires. You may leave the
microcontroller attached to the breadboard.

\subsubsection{Attach the microcontroller to the breadboard}
If it's not already, carefully insert the pins at the bottom of your microcontroller into the breadboard. Refer back
to \ref{pinout} for pin labels. When placing the board into the breadboard, make sure that the microcontroller is oriented such that:
\begin{itemize}
    \item The pin labeled \textbf{5V} is inserted in hole at \textbf{Column H, Row 1} of the breadboard (or \textbf{H1}, for short)
    \item The pin labeled \textbf{GPIO2} is inserted in hole \textbf{D1} of the breadboard
    \item The pin labeled \textbf{GPIO20} is inserted in hole \textbf{H7} of the breadboard
    \item the pin labeled \textbf{GPIO21} is inserted in hole \textbf{D7} of the breadboard
\end{itemize}
You may need to apply more pressure than expected to seat the microcontroller properly in the breadboard. When its over, it should look like this:
\begin{figure}[H]
    \centering
    \includegraphics[width=.6\linewidth]{common/microcontroller_seated_in_breadboard.jpg}
    \caption{So far, so good!}
\end{figure}

\subsubsection{Connect the necessary jumper wires}
\begin{itemize}
    \item Using a jumper wire, place one end of the wire into hole \textbf{J3} of the breadboard and the other end in
    hole \textbf{D16} of the breadboard. This will provide \textbf{3.3} volts of power to the temperature/humidity sensor.
    \item Using another jumper wire, place one end of the wire into hole \textbf{J2} of the breadboard and the other
    end in hole \textbf{D17} of the breadboard. This will provide the ground connection for the temperature/humidity sensor.
    \item Using a 3rd jumper wire, place one end of the wire into hole \textbf{B6} of the breadboard and the other
    end in hole \textbf{D18} of the breadboard. This will provide a clock signal to the sensor board.
    \item Finally, using a 4th jumpre wire, place one end of the wire into hole \textbf{B5} of the breadboard and the other
    end in hole \textbf{D19} of the breadboard. This will transmit data from the sensor back to the microcontroller.
\end{itemize}

You should be left with something that looks like this:
\begin{figure}[H]
    \centering
    \includegraphics[width=.6\linewidth]{chapter_4/sensor_board_wired.jpg}
    \caption{There are 4 wires that will be connected between the microcontroller and the sensor board.}
\end{figure}

\subsubsection{Attach the temperature/humidity sensor to the breadboard}
Plug the 4 pins of the sensor board into the breadboard just under where the 4 jumper wires are lined up. Make sure the pins
of the board are lined up with the jumper wires and the board points away from them. It should look like this:

\begin{figure}[H]
    \centering
    \includegraphics[width=.6\linewidth]{chapter_4/sensor_board_connected.jpg}
    \caption{Click the button highlighted in red.}
\end{figure}

\subsection{Programming the microcontroller}
Once all of the wiring is correct, connect the USB cable to the microcontroller and go to https://viper-ide.org/ in your
computer's web browser. Click on the USB icon in the top right and choose your microcontroller from the list:

\begin{figure}[H]
    \centering
    \includegraphics[width=.6\linewidth]{common/viper_ide_usb_connect.png}
    \caption{Click the button highlighted in red.}
\end{figure}

If you see multiple items in the dialog that pops up, choose the one that starts with "USB JTAG". See below for an example:
\begin{figure}[H]
    \centering
    \includegraphics[width=.6\linewidth]{common/viper_ide_usb_choice_connect.png}
    \caption{Click the button highlighted in red.}
\end{figure}

Once you have connected, you will see a green dialog labeled "Device connected" and the file manager on the list
will populate with the list of files installed on the device:
\begin{figure}[H]
    \centering
    \includegraphics[width=.6\linewidth]{common/viper_connected.png}
    \caption{Note: update this image with the final file list}
\end{figure}

Click on the file named "chapter\_4a.py". This will load the code in the editor for this section. Read through the comments
and the code to get a sense for how it works. Once you are ready, you can click the blue play button in the upper left of the window to start the script:
\begin{figure}[H]
    \centering
    \includegraphics[width=.6\linewidth]{chapter_4/play_chapter_4a.png}
    \caption{Once started, this script will run forever and output values into the terminal window. You can stop by pressing the red stop button.}
\end{figure}

\subsection{Making it fancier}
Having the temperature and humidity print out to the terminal is pretty cool (or pretty warm depending on where you are).
But wouldn't it be better if we didn't have to be connected to a computer to see the values? We can give our device a screen and have it print the values out there as well.

\subsubsection{Connect the necessary jumper wires}
\begin{itemize}
    \item First, disconnect the USB cable from your microcontroller. Doing this prevents accidentally connecting power to somewhere it shouln't go!
    \item Using a jumper wire, place one end of the wire into hole \textbf{E16} of the breadboard and the other end in
    hole \textbf{E25} of the breadboard. This will provide \textbf{3.3} volts of power to the screen.
    \item Using another jumper wire, place one end of the wire into hole \textbf{E17} of the breadboard and the other end
    in hole \textbf{E24} of the breadboard. This will provide the ground connection for the screen.
    \item Using a 3rd jumper wire, place one end of the wire into hole \textbf{E18} of the breadboard and the other end
    in hole \textbf{E26} of the breadboard. This will provide a clock signal to the screen.
    \item Finally, using a 4th jumpre wire, place one end of the wire into hole \textbf{B4} of the breadboard and the other
    end in hole \textbf{E27} of the breadboard. This will transmit data from the microcontroller to the screen to be displayed.
\end{itemize}

You should be left with something that looks like this:
\begin{figure}[H]
    \centering
    \includegraphics[width=.6\linewidth]{chapter_4/screen_wired.jpg}
    \caption{The first 3 wires are connected right behind the wires for the sensor. The last wire is connected to the microcontroller.}
\end{figure}

\subsubsection{Attach the screen to the breadboard}
Plug the 4 pins of the screen into the breadboard just under where the 4 new jumper wires are lined up. Make sure the pins of
the screen are lined up with the jumper wires and the screen points away from them. It should look like this:

\begin{figure}[H]
    \centering
    \includegraphics[width=.6\linewidth]{chapter_4/screen_connected.jpg}
    \caption{Click the button highlighted in red.}
\end{figure}

Click on the file named "chapter\_4b.py". This will load the code in the editor for this section. Read through the comments and
the code to get a sense for how it works. Once you are ready, you can click the blue play button in the upper left of the window to start the script:

\begin{figure}[H]
    \centering
    \includegraphics[width=.6\linewidth]{chapter_4/play_chapter_4b.png}
    \caption{Once started, this script will run forever and output values into the terminal window and to the OLED screen.
    You can stop by pressing the red stop button.}
\end{figure}

\section{Review}
% go over what we have accomplished (maybe go into more detail about the circuit and the pins on the microcontroller?

\section{Possible Extensions}
If you want to do some experimentation, try these:

\begin{itemize}
    \item Connect an LED and have it light up when the temperature gets warmer than some threshold
    \item Connect a button and have the screen change between several different readouts when pressed
\end{itemize}

\chapter{Project 5: Game}
% chapter contents
\chapter{Project 6: Chat}
% chapter contents
\chapter{Project 7: Sound}
% chapter contents

\appendix
\chapter{Electronics Essentials}
% chapter contents
\chapter{Python Primer}
If you're new to Python, this section will give you a few things you should know
in order to better understand the projects in this guide. This is by no means a
complete or comprehensive look at the Python language. For that, we recommend looking
at the official Python site and reading through the \href{https://docs.python.org/3/tutorial/}{tutorial}
there.

\begin{tcolorbox}
    Note: for the projects being used here, we are using an implementation of
    Python known as \href{https://micropython.org/}{MicroPython}. This version
    is meant to run on microcontrollers with limited resources. It
    also has built into it libraries for dealing with hardware devices that are
    not part of the standard CPython distribution. Therefore, not all Python examples
    you find online will run on your microcontroller and not all projects for a
    microcontroller can be run on your computer. But a lot of the code can be shared
    so the lessons you learn here can apply to other Python projects.
\end{tcolorbox}

Here is a sample of a small Python script. We will disect and explain what each
section does below:

\begin{lstlisting}[language=Python,caption=An example Python script]
def show(message, repeat=1):
    """This function prints the given message to the
    console as smany times as specified in the
    srepeat parameter.
    """

    for iteration in range(0, repeat):
        print(iteration, message)

name = input("What is your name: ")
show(name)
show(name, repeat=3)
\end{lstlisting}

On line 1, we are defining a function named show. This function accepts two parameters,
message and repeat. The message parameter is required and the repeat parameter
is optional with a default value of 1.

Lines 2 through 5 comprise the docstring for the function. This information is meant
for programmers to read and explains what the function does. It does not affect how
the function works.

Line 7 starts a loop. The loop will repeat the statements in the loop body until
a condition is met. In this case, it will loop until it has performed the operation
for each repeat.

Line 8 is the body of the loop. This statement will print the message that the user
passed in to the console along with the iteration number of the loop.

Line 10 prompts the user for their name and saves the result in a variable called
name.

Line 11 calls our show function which will print the user's name once (the default).

Line 12 calls our show function again, this time saying that we want to repeat the
loop of printing the name twice.\newline

Running the program, we will see output like this:
\begin{verbatim}
$ python program.py
What is your name: Emily
0 Emily
0 Emily
1 Emily
2 Emily
$
\end{verbatim}

Another feature that you'll see used often in Python are classes. Classes are a
convienent way to model something in your program that holds state and implements
functionality. For example, let's say that we are writing a game about racing go-
karts. We need to allow each player to have their own kart and keep track of how
fast it is going, which way they are turning, and allow the kart to speed up and
slow down. Here is a small class that will help us do that:

\begin{lstlisting}[language=Python,caption=An example of a Python class]
class Kart:
    MAXIMUM_SPEED = 100

    def __init__(self):
        """The kart starts motionless at the beginning"""
        self._speed = 0
        self._direction = 0
        self._acceleration = 0

    def brake(self):
        """This is called when the user presses the brake button"""
        self._acceleration = -5

    def accelerate(self):
        """This is called when the user presses the accelerator button"""
        self._acceleration = 5

    def steer(self, direction):
        """This is called when the user presses left or right"""
        self._direction = direction

    def update(self, ticks):
        """Update will be called by our game engine and will be
        provided the number of ticks since it was last called.
        """

        self._speed += self._acceleration * ticks

        # limit our speed so that we don't go faster than our
        # kart is allowed to, or slower than 0
        if self._speed > Kart.MAXIMUM_SPEED:
            self._speed = Kart.MAXIMUM_SPEED
        if self._speed < 0:
            self._speed = 0
\end{lstlisting}

Looking at this class, there are 5 methods. The first one (on line 4) is a special
method that is called by the Python interpreter whenever a new Kart is created. It
will initialize some variables for this particular Kart object.

You may have noticed that the first method takes a parameter called "self". This
is the first parameter of all methods in a class in Python. It is automatically passed
by the interpreter and is a reference to the current object. It lets us access the
variables that belong to the class, like those we defined in the \_\_init\_\_ method.

Speaking of the variables in the \_\_init\_\_ method. Notice how we named them all with
an underscore? This is a convention in Python that says they are private to our class
and that code written outside of the class shouldn't access them directly. That means
that our class should provide ways to modify or read these variables via other methods.

The second method starts on line 10. This is called when the player presses the
brake button on their controller and will set our Kart's acceleration to a negative
value so that we start to slow down. It modifies the private \_acceleration member of
the class.

The third starts on line 14. It is the opposite of braking and will start speeding
our Kart up when the user presses the accelerator. It aslo modifies the private
\_acceleration member of the class.

The fourth method, line 18, is again something to deal with user input. This time
we can see that it takes a second parameter, direction. If the user presses left on
their controller, then we can expect left to be passed here. The same for right. We
will modify the private \_direction member here.

Finally, we have a fifth method starting on line 22. This method is called by our
game engine and uses the class members to determine what happens to the Kart throughout
the game. That is, it is asking the Kart to update itself at a certain moment in time
(usually once per frame) so that next time it draws it to the screen, it will be
in the updated location.

Notice in the last method, we are accessing not only our own variables, \_speed, and
\_acceleration, but we are also reading a class variable, Kart.MAXIMUM\_SPEED. Unlike
our member variables, a class variable is the same for all instances of a class. It
is useful here to keep the game fair so that all Karts have the same limitation on
their speed.

\end{document}
